% This syllabus template was created by:
% Brian R. Hall
% Assistant Professor, Champlain College
% www.brianrhall.net

% Document settings
\documentclass[11pt]{article}
\usepackage[margin=1in]{geometry}
\usepackage[pdftex]{graphicx}
\usepackage{multirow}
\usepackage{setspace}
\usepackage{hyperref}
\hypersetup{
    colorlinks = true
    }
\pagestyle{plain}
\setlength\parindent{0pt}
\setlength{\parskip}{1em}



\begin{document}

% Course information
\begin{center} 
\includegraphics[height=1in, keepaspectratio]{LogoNew.png}
\end{center}
%\vspace{2mm}

% Professor information
\begin{center} 
\begin{tabular}{ c c }
  \multirow{6}{*}%{\includegraphics[height=1.25in,width=1in]{pic_blank.png}} 
        & \\\\
&  \LARGE Bayesian Modeling for Socio-Environmental Data\\\\
  & \LARGE August 15-25, 2017 \\\\
& \large Instructors: Chris Che-Castaldo, Mary B. Collins, N. Thompson Hobbs \\\\
  %& \large website \\
%  & \large 113 Marshall Hall \\
%  & \large Office Hours: Wednesday 1-3pm \\
%  & \large (315) 470-6538 \\
%  &\\
%  & \large Teaching Assistant: Michael Dominic Persson\\\\
%  & \large mdpersso@syr.edu \\
%    & \large Office Hours: Monday Moon Library Open Table 1240-2pm 
\vspace{2mm}
\end{tabular}
%\vspace{5mm}

\emph{Course syllabus is subject to change}. \\
\end{center}

% Course details
\textbf{\large Course Preparation:} 
R is a crucial skill for success in this course. Considering reviewing Tom's R Primer, and head over to Quick R, \href{http://www.statmethods.net/index.html}{http://www.statmethods.net/index.html}, which gives a nice overview of basic R functionality. We ask that you are fluent with the following topics from Quick-R:
\begin{enumerate}
\item Data Types, Importing Data (from excel or a .csv file), Keyboard Input, and Missing Values under Data Input
\item All topics under Data Management
\item All topics under Graphs
\end{enumerate}
Specifically, you should be very comfortable with manipulating matrices and lists and writing and using custom functions.

We strongly recommend purchasing Hobbs \& Hooten 2015 \href{https://www.amazon.com/Bayesian-Models-Statistical-Primer-Ecologists/dp/0691159289}{https://www.amazon.com/Bayesian-Models-Statistical-Primer-Ecologists/dp/0691159289}. The first three chapters provide foundational material that we will cover fairly quickly in the course, so if you have not had a course in mathematical statistics, reading those chapters before the course is crucial. The structure of the course closely follows the organization of the book.  It will be a useful reference after the course.

\textbf{\large Course Logistics:} 
We will be starting on the first day at 8am.  Every other day we will begin at 9am. We will usually end at 5 or 5:30.  

Remember that lunch will be served at SESYNC each day during the course.

Course materials will be distributed throughout the course via \href{https://github.com/CCheCastaldo/SESYNCBayes}{GitHub}.  Clone our repo and plan on pulling down new material often.

\newpage
\begin{center}
\textbf{\large Course Schedule}
\end{center}

\begin{enumerate}
%%%%%%Day 1
\item[\textbf{Day 1:}] Probability (\emph{review H\&H Chs 1-3})

\begin{itemize}
\item Participant and Course Introduction %[notes: TOM LEAD (no lab)]
\item \emph{What Sets Bayes Apart?} %[notes: TOM LEAD justification why we need laws of prob]
\item \emph{Laws of Probability} \& Probability Lab Exercises: I-VI %[notes: MARY LEAD conditioning/independence, tot prob, chain rule]
\item \emph{Probability Distributions and Moment Matching} \& Probability Lab Exercise VII-VIII %[notes: CHRIS LEAD support, pmf/pdf, cdf, quantile function, cheat sheets]
\end{itemize}

%%%%%%Day 2
\item[\textbf{Day 2:}] Likelihood \& Bayes' Theorem (\emph{review H\&H Chs 4-6})
\begin{itemize}
\item \emph{Probability Distributions and Moment Matching} \& Probability Lab Exercise IX %[notes: CHRIS LEAD support, pmf/pdf, cdf, quantile function, cheat sheets]
\item \emph{Likelihood}  %[notes: CHRIS LEAD profile, total likelihood, max likelihood, ratios]
\item \emph{Bayes' Theorem}  \& Bayes' Theorem Exercises  %[notes: TOM LEAD]
\item Happy Hour at SESYNC 
\end{itemize}

%%%%%%Day 3
\item [\textbf{Day 3:}]  \emph{Introduction to Hierarchical Models} \& Hierarchical Modeling Board Work  %[notes: TOM LEAD]

%%%%%%Day 4
\item[\textbf{Day 4:}] MCMC \& JAGS (\emph{review H\&H Chps 8})
\begin{itemize}
\item \emph{Conjugate Priors} \& Conjugate Priors Brief Exercise 
\item \emph{MCMC} 
\item JAGS Primer Work \emph{and/or} MCMC I and II lab
\end{itemize}

%%%%%%Day 5
\item[\textbf{Day 5:}] MCMC \& JAGS (\emph{review H\&H Chps 8})
\begin{itemize}
\item JAGS Primer Work \emph{and/or} MCMC I and II lab
\item \emph{Inference From a Single Model}  %[notes: TOM LEAD]
\item \emph{Vague Priors in Non-Linear Models} \& Islands Problem Set  %[notes: TOM LEAD] 
\end{itemize}

\item[\textbf{Day 7:}] Bayesian Regression, Analysis of Covariance, \& Multi-Level Modeling (\emph{review H\&H Chp 9-12})
\begin{itemize}
\item \emph{Bayesian Regression} \& Practice: Writing Models and Psuedo Coding %[notes: CHRIS/TOM LEAD]
\item \emph{Multi-Level Modeling} \& Begin Multi-Level Modeling Lab  %[notes: TOM LEAD] 
\end{itemize}

%\newpage

%%%%%%Day 8
\item[\textbf{Day 8:}] Multi-Level Modeling Continued, Model Checking, \& Model Selection

\begin{itemize}
\item  Continue Multi-Level Modeling Lab % [notes: CHRIS/TOM LEAD]
\item \emph{Model Checking}% [notes: TOM LEAD] 
\item \emph{Model Selection}% [notes: TOM LEAD]
\end{itemize}

%%%%%%Day 9
\item[\textbf{Day 9:}] Flavors of Modeling

\begin{itemize}
\item \emph{Mixture Models, Zero Inflation, \& Occupancy} \& Swiss Birds Lab %[notes: TOM LEAD]
\item \emph{Dynamic Models} \& Lynx Exercises %[notes: TOM LEAD]
\end{itemize}

\newpage

%%%%%%Day 10
\item[\textbf{Day 10:}] Spatial Modeling

\begin{itemize}
\item \emph{Continuous Spatial Process Modeling} \& Source Diversity Lab% [notes: TOM/MARY LEAD]
%\item \emph{Synthesis via Meta Analysis} \& Pediatric Deaths Lab %[notes: TOM LEAD] 
\end{itemize}

%%%%%%Day 11
\item[\textbf{Day 11:}] Using Your Bayes' Skills

\begin{itemize}
\item Participant Project Presentations and Feedback 
\end{itemize}

\end{enumerate}




\end{document}



