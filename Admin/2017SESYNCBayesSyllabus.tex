% This syllabus template was created by:
% Brian R. Hall
% Assistant Professor, Champlain College
% www.brianrhall.net

% Document settings
\documentclass[11pt]{article}
\usepackage[margin=1in]{geometry}
\usepackage[pdftex]{graphicx}
\usepackage{multirow}
\usepackage{setspace}
\usepackage{hyperref}
\hypersetup{
    colorlinks = true
    }
\pagestyle{plain}
\setlength\parindent{0pt}
\setlength{\parskip}{1em}



\begin{document}

% Course information
\begin{center} 
\includegraphics[height=1in, keepaspectratio]{LogoNew.png}
\end{center}
%\vspace{2mm}

% Professor information
\begin{center} 
\begin{tabular}{ c c }
  \multirow{6}{*}%{\includegraphics[height=1.25in,width=1in]{pic_blank.png}} 
        & \\\\
&  \LARGE Bayesian Modeling for Socio-Environmental Data\\\\
  & \LARGE August 15-25, 2017 \\\\
& \large Instructors: Chris Che-Castaldo, Mary B. Collins, N. Thompson Hobbs \\\\
  %& \large website \\
%  & \large 113 Marshall Hall \\
%  & \large Office Hours: Wednesday 1-3pm \\
%  & \large (315) 470-6538 \\
%  &\\
%  & \large Teaching Assistant: Michael Dominic Persson\\\\
%  & \large mdpersso@syr.edu \\
%    & \large Office Hours: Monday Moon Library Open Table 1240-2pm 
\vspace{2mm}
\end{tabular}
%\vspace{5mm}

\emph{Course syllabus is subject to change}. \\
\end{center}

% Course details
\textbf{\large Course Description:} 
R is a crucial skill for success in this course. Considering reviewing Tom's R Primer, and head over to Quick R, \href{http://www.statmethods.net/index.html}{http://www.statmethods.net/index.html}, which gives a nice overview of basic R functionality. We ask that you are fluent with the following topics from Quick-R:
\begin{enumerate}
\item Data Types, Importing Data (from excel or a .csv file), Keyboard Input, and Missing Values under Data Input
\item All topics under Data Management
\item All topics under Graphs
\end{enumerate}
Specifically, you should be very comfortable with manipulating matrices and lists and writing and using custom functions.

We strongly recommend purchasing Hobbs \& Hooten 2015 \href{https://www.amazon.com/Bayesian-Models-Statistical-Primer-Ecologists/dp/0691159289}{https://www.amazon.com/Bayesian-Models-Statistical-Primer-Ecologists/dp/0691159289}. The first three chapters provide foundational material that we will cover fairly quickly in the course, so if you have not had a course in mathematical statistics, reading those chapters before the course is crucial. The structure of the course closely follows the organization of the book.  It will be a useful reference after the course.

We will be starting on the first day at 8am.  Every other day we will begin at 9am. We will usually end at 5 or 5:30.  

Remember that lunch will be served at SESYNC each day during the course.

Course materials will be distributed throughout the course via \href{https://github.com/CCheCastaldo/SESYNCBayes}{GitHub}.  Clone our repo and plan on pulling down new material often.

\newpage
\begin{center}
\textbf{\large Course Schedule}
\end{center}

\begin{enumerate}
%%%%%%Day 1
\item[\textbf{Day 1:}] Probability

\begin{itemize}
\item Participant and Course Introduction %TOM LEAD (no lab)
\item \emph{What Sets Bayes Apart?} %TOM LEAD justification why we need laws of prob
\item \emph{Laws of Probability} \& Probability Lab Exercises: I-V %MARY LEAD conditioning/independence, tot prob, chain rule
\item \emph{Probability Concepts and Notation} \& Probability Lab Exercise VI %CHRIS LEAD support, pmf/pdf, cdf, quantile function, cheat sheets
\item \emph{Probability Distributions}  \& Probability Lab Exercise VII %TOM LEAD moment matching
\end{itemize}

%%%%%%Day 2
\item[\textbf{Day 2:}] Likelihood \& Bayes' Theorem

\begin{itemize}
\item \emph{Likelihood} \& Likelihood Problem Set %CHRIS LEAD profile, total likelihood, max likelihood, ratios
\item \emph{Bayes' Theorem Part A} \& Bayes' Theorem Exercises I-IV %TOM LEAD plotting data, prior on theta
\item \emph{Bayes' Theorem Part B} \& Bayes' Theorem Exercises  V-VI %TOM LEAD Likelihood and joint
\item \emph{Bayes' Theorem Part C} \& Bayes' Theorem Exercises VII-IX, XIV, XVI, XVII %TOM LEAD
\item \emph{Introduction to Hierarchical Models} \& Hierarchical Modeling Board Work %TOM LEAD
\end{itemize}


%%%%%%Day 3
\item[\textbf{Day 3:}] Priors \& Markov chain Monte Carlo

\begin{itemize}
\item \emph{Conjugate Priors Part A} \& Beta-Binomial Exercise %MARY LEAD Bayes lab problem 15?
\item \emph{Conjugate Priors Part B} \& Poisson-Gamma Exercise %MARY LEAD 
\item \emph{Normal-Normal and Normal-Inverse Gamma Usage} \& Normal-Normal and Normal-Inverse Gamma Exercise %TOM LEAD 
\item \emph{MCMC Overview} \& MCMC Exercise III %TOM LEAD 
\item \emph{MCMC Next Steps} \& MCMC Exercises IV-V %TOM LEAD 
\item \emph{MCMC Accept Reject} \& MCMC Metropolis-Hastings Coding Exercise  %TOM LEAD 
\item \emph{Happy Hour at SESYNC} 
\end{itemize}

%%%%%%Day 4
\item[\textbf{Day 4:}] JAGS \& Modeling Practice

\begin{itemize}
\item JAGS Primer Work
\item \emph{Inference From a Single Model} %TOM LEAD 
\item \emph{Vague Priors in Non-Linear Models} \& Islands Problem Set %TOM LEAD 
\item Non-Linear Priors Problem Set %TOM LEAD 
\end{itemize}

%%%%%%Day 5
\item[\textbf{Day 5:}] Bayesian Regression, Analysis of Covariance, \& Multi-Level Modeling

\begin{itemize}
\item \emph{Bayesian Regression} \& Practice: Writing Models and Psuedo Coding %CHRIS/TOM LEAD 
\item \emph{Multi-Level Modeling} \& Begin Multi-Level Modeling Lab  %TOM LEAD 
\end{itemize}

\newpage

%%%%%%Day 6
\item[\textbf{Day 6:}] Multi-Level Modeling Continued, Model Checking, \& Model Selection

\begin{itemize}
\item  Continue Multi-Level Modeling Lab  %CHRIS/TOM LEAD 
\item \emph{Model Selection} %TOM LEAD 
\item \emph{Model Checking} %TOM LEAD 
\item \emph{Utility and Use of the Half-Couchy Prior} %TOM LEAD
\end{itemize}

%%%%%%Day 7
\item[\textbf{Day 7:}] Flavors of Modeling

\begin{itemize}
\item \emph{Mixture Models, Zero Inflation, \& Occupancy} \& Swiss Birds Lab %TOM LEAD 
\end{itemize}

%%%%%%Day 8
\item[\textbf{Day 8:}] Dynamic Models \& Applying Your Knowledge

\begin{itemize}
\item \emph{Dynamic Models} \& Lynx Exercises %TOM LEAD 
\item Participant Project Work
\end{itemize}

%%%%%%Day 9
\item[\textbf{Day 9:}] Spatial Modeling

\begin{itemize}
\item \emph{Continuous Spatial Process Modeling} \& Source Diversity Lab %TOM LEAD 
\item \emph{Synthesis via Meta Analysis} \& Pediatric Deaths Lab %TOM LEAD 
\end{itemize}

%%%%%%Day 10
\item[\textbf{Day 10:}] Using Your Bayes' Skills

\begin{itemize}
\item Participant Project Presentations and Feedback 
\end{itemize}

\end{enumerate}




\end{document}



